\documentclass[journal,12pt,twocolumn]{IEEEtran}
  \usepackage{setspace}
  \usepackage{gensymb}
  \usepackage{graphicx}
  \singlespacing
 \graphicspath{ {/user/adarshsrivastava/desktop/Matrix Theory/Assignemnt_7} }
  \usepackage[cmex10]{amsmath}
  \usepackage{amsthm}
  \usepackage{hyperref}
  \usepackage{mathrsfs}
  \usepackage{txfonts}
  \usepackage{stfloats}
  \usepackage{bm}
  \usepackage{cite}
  \usepackage{cases}
  \usepackage{subfig}
  \usepackage{longtable}
  \usepackage{multirow}
  \usepackage{enumitem}
  \usepackage{mathtools}
  \usepackage{steinmetz}
  \usepackage{tikz}
  \usepackage{circuitikz}
  \usepackage{verbatim}
  \usepackage{tfrupee}
  \usepackage[breaklinks=true]{hyperref}
  \usepackage{tkz-euclide}
  \usetikzlibrary{calc,math}
  \usepackage{listings}
      \usepackage{color}                                            %%
      \usepackage{array}                                            %%
      \usepackage{longtable}                                        %%
      \usepackage{calc}                                             %%
      \usepackage{multirow}                                         %%
      \usepackage{hhline}                                           %%
      \usepackage{ifthen}                                           %%
      \usepackage{lscape}     
  \usepackage{multicol}
  \usepackage{chngcntr}
  \DeclareMathOperator*{\Res}{Res}
  \renewcommand\thesection{\arabic{section}}
  \renewcommand\thesubsection{\thesection.\arabic{subsection}}
  \renewcommand\thesubsubsection{\thesubsection.\arabic{subsubsection}}
  \renewcommand\thesectiondis{\arabic{section}}
  \renewcommand\thesubsectiondis{\thesectiondis.\arabic{subsection}}
  \renewcommand\thesubsubsectiondis{\thesubsectiondis.\arabic{subsubsection}}
  \hyphenation{op-tical net-works semi-conduc-tor}
  \def\inputGnumericTable{}                                 %%
  \lstset{
  %language=C,
  frame=single, 
  breaklines=true,
  columns=fullflexible
  }
  \begin{document}
  \newtheorem{theorem}{Theorem}[section]
  \newtheorem{problem}{Problem}
  \newtheorem{proposition}{Proposition}[section]
  \newtheorem{lemma}{Lemma}[section]
  \newtheorem{corollary}[theorem]{Corollary}
  \newtheorem{example}{Example}[section]
  \newtheorem{definition}[problem]{Definition}
  \newcommand{\BEQA}{\begin{eqnarray}}
  \newcommand{\EEQA}{\end{eqnarray}}
  \newcommand{\define}{\stackrel{\triangle}{=}}
  \bibliographystyle{IEEEtran}
  \providecommand{\mbf}{\mathbf}
  \providecommand{\pr}[1]{\ensuremath{\Pr\left(#1\right)}}
  \providecommand{\qfunc}[1]{\ensuremath{Q\left(#1\right)}}
  \providecommand{\sbrak}[1]{\ensuremath{{}\left[#1\right]}}
  \providecommand{\lsbrak}[1]{\ensuremath{{}\left[#1\right.}}
  \providecommand{\rsbrak}[1]{\ensuremath{{}\left.#1\right]}}
  \providecommand{\brak}[1]{\ensuremath{\left(#1\right)}}
  \providecommand{\lbrak}[1]{\ensuremath{\left(#1\right.}}
  \providecommand{\rbrak}[1]{\ensuremath{\left.#1\right)}}
  \providecommand{\cbrak}[1]{\ensuremath{\left\{#1\right\}}}
  \providecommand{\lcbrak}[1]{\ensuremath{\left\{#1\right.}}
  \providecommand{\rcbrak}[1]{\ensuremath{\left.#1\right\}}}
  \theoremstyle{remark}
  \newtheorem{rem}{Remark}
  \newcommand{\sgn}{\mathop{\mathrm{sgn}}}
  \providecommand{\abs}[1]{\left\vert#1\right\vert}
  \providecommand{\res}[1]{\Res\displaylimits_{#1}} 
  \providecommand{\norm}[1]{\left\lVert#1\right\rVert}
  %\providecommand{\norm}[1]{\lVert#1\rVert}
  \providecommand{\mtx}[1]{\mathbf{#1}}
  \providecommand{\mean}[1]{E\left[ #1 \right]}
  \providecommand{\fourier}{\overset{\mathcal{F}}{ \rightleftharpoons}}
  %\providecommand{\hilbert}{\overset{\mathcal{H}}{ \rightleftharpoons}}
  \providecommand{\system}{\overset{\mathcal{H}}{ \longleftrightarrow}}
  	%\newcommand{\solution}[2]{\textbf{Solution:}{#1}}
  \newcommand{\solution}{\noindent \textbf{Solution: }}
  \newcommand{\cosec}{\,\text{cosec}\,}
  \providecommand{\dec}[2]{\ensuremath{\overset{#1}{\underset{#2}{\gtrless}}}}
  \newcommand{\myvec}[1]{\ensuremath{\begin{pmatrix}#1\end{pmatrix}}}
  \newcommand{\mydet}[1]{\ensuremath{\begin{vmatrix}#1\end{vmatrix}}}
  \numberwithin{equation}{subsection}
  \makeatletter
  \@addtoreset{figure}{problem}
  \makeatother
  \let\StandardTheFigure\thefigure
  \let\vec\mathbf
  \renewcommand{\thefigure}{\theproblem}
  \def\putbox#1#2#3{\makebox[0in][l]{\makebox[#1][l]{}\raisebox{\baselineskip}[0in][0in]{\raisebox{#2}[0in][0in]{#3}}}}
       \def\rightbox#1{\makebox[0in][r]{#1}}
       \def\centbox#1{\makebox[0in]{#1}}
       \def\topbox#1{\raisebox{-\baselineskip}[0in][0in]{#1}}
       \def\midbox#1{\raisebox{-0.5\baselineskip}[0in][0in]{#1}}
  \vspace{3cm}
  \title{Assignment 7}
  \author{Rubeena Aafreen}
  \maketitle
  \newpage
  \bigskip
  %\renewcommand{\thefigure}{\theenumi}
  \renewcommand{\thetable}{\theenumi}
  \begin{center}
{\underline{\Large \bf Row Reduced Echelon Form}}
\end{center}
\begin{abstract}
This document solves problem based on solution of system of linear equations using Row Reduction
\end{abstract}
%
Download all solutions from 
\begin{lstlisting}
https://github.com/rubeenaafreen20/EE5609/tree/master/Assignment7
\end{lstlisting}
%
\vspace{-5mm}
\section{Problem}
Find all solutions of
 \begin{align}
 2x_1-3x_2-7x_3+5x_4+2x_5=-2\\x_1-2x_2-4x_3+3x_4+x_5=-2\\2x_1-4x_3+2x_4+x_5=3\\x_1-5x_2-7x_3+6x_4+2x_5=-7
 \end{align}
\section{Solution}
   The given equations can be written as,
   \begin{align}
  \myvec{2&-3&-7&5&2\\1&-2&-4&3&1\\2&0&-4&2&1\\1&-5&-7&6&2}\vec{x}=\myvec{-2\\-2\\3\\7}
   \end{align}
   Now, we form the augmented matrix and perform Row reduction,
   \begin{align}
      \myvec{2&-3&-7&5&2&\vrule&-2\\1&-2&-4&3&1&\vrule&-2\\2&0&-4&2&1&\vrule&3\\1&-5&-7&6&2&\vrule&7}\\
        \xleftrightarrow{R_3=R_3-R_1}\myvec{2&-3&-7&5&2&\vrule&-2\\1&-2&-4&3&1&\vrule&-2\\0&3&3&-3&-1&\vrule&5\\1&-5&-7&6&2&\vrule&7}\\
     \xleftrightarrow{R_1=\frac{1}{2}R_1}\myvec{1&\frac{-3}{2}&\frac{-7}{2}&\frac{5}{2}&1&\vrule&-1\\1&-2&-4&3&1&\vrule&-2\\0&3&3&-3&-1&\vrule&5\\1&-5&-7&6&2&\vrule&7}\\
      \xleftrightarrow{R_2=R_2-R_1,R_4=R_4-R_1}\myvec{1&\frac{-3}{2}&\frac{-7}{2}&\frac{5}{2}&1&\vrule&-1\\0&-\frac{1}{2}&-\frac{1}{2}&\frac{1}{2}&0&\vrule&-1\\0&3&3&-3&-1&\vrule&5\\0&-\frac{7}{2}&-\frac{7}{2}&\frac{7}{2}&1&\vrule&-6}\\
      \xleftrightarrow{R_1=R_1-3R_2}\myvec{1&0&-2&1&1&\vrule&2\\0&-\frac{1}{2}&-\frac{1}{2}&\frac{1}{2}&0&\vrule&-1\\0&3&3&-3&-1&\vrule&5\\0&-\frac{7}{2}&-\frac{7}{2}&\frac{7}{2}&1&\vrule&-6}\\
      \xleftrightarrow{R_3=R_3+6R_2, R_4=R_4-7R_2}\myvec{1&0&-2&1&1&\vrule&2\\0&-\frac{1}{2}&-\frac{1}{2}&\frac{1}{2}&0&\vrule&-1\\0&0&0&0&-1&\vrule&-1\\0&0&0&0&1&\vrule&1}\\
      \xleftrightarrow{R_2=-2R_2}\myvec{1&0&-2&1&1&\vrule&2\\0&1&1&-1&0&\vrule&2\\0&0&0&0&-1&\vrule&-1\\0&0&0&0&1&\vrule&1}\\
      \xleftrightarrow{R_1=R_1+R_3,R_4=R_4+R_3,R_3=-R_3}\myvec{1&0&-2&1&0&\vrule&1\\0&1&1&-1&0&\vrule&2\\0&0&0&0&1&\vrule&1\\0&0&0&0&0&\vrule&0}
\end{align}
So,
\begin{align}
x_1-2x_3+x_4=1\\
x_2+x_3-x_4=2\\
x_5=1
\end{align}
Solving the equations we get,
\begin{align}
x_1=1+2x_3-x_4\\
x_2=2-x_3+x_4\\
x_5=1
\end{align}
which can be written as,
   \begin{align}
   \vec{x} = \myvec{x_1\\x_2\\x_3 \\ x_4 \\ x_5} \\
   \implies \vec{x}=\myvec{1+2x_3-x_4\\2-x_3+x_4\\x_3\\x_4\\1}
\end{align}
where $x_3$, $x_4$ $\in$ $\mathbb{R}$.
\end{document}