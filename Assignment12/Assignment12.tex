\documentclass[journal,12pt]{IEEEtran}
\usepackage{longtable}
\usepackage{setspace}
\usepackage{gensymb}
\singlespacing
\usepackage[cmex10]{amsmath}
\newcommand\myemptypage{
	\null
	\thispagestyle{empty}
	\addtocounter{page}{-1}
	\newpage
}
\usepackage{amsthm}
\usepackage{mdframed}
\usepackage{mathrsfs}
\usepackage{txfonts}
\usepackage{stfloats}
\usepackage{bm}
\usepackage{cite}
\usepackage{cases}
\usepackage{subfig}

\usepackage{longtable}
\usepackage{multirow}

\usepackage{enumitem}
\usepackage{mathtools}
\usepackage{steinmetz}
\usepackage{tikz}
\usepackage{circuitikz}
\usepackage{verbatim}
\usepackage{tfrupee}
\usepackage[breaklinks=true]{hyperref}
\usepackage{graphicx}
\usepackage{tkz-euclide}

\usetikzlibrary{calc,math}
\usepackage{listings}
    \usepackage{color}                                            %%
    \usepackage{array}                                            %%
    \usepackage{longtable}                                        %%
    \usepackage{calc}                                             %%
    \usepackage{multirow}                                         %%
    \usepackage{hhline}                                           %%
    \usepackage{ifthen}                                           %%
    \usepackage{lscape}     
\usepackage{multicol}
\usepackage{chngcntr}

\DeclareMathOperator*{\Res}{Res}

\renewcommand\thesection{\arabic{section}}
\renewcommand\thesubsection{\thesection.\arabic{subsection}}
\renewcommand\thesubsubsection{\thesubsection.\arabic{subsubsection}}

\renewcommand\thesectiondis{\arabic{section}}
\renewcommand\thesubsectiondis{\thesectiondis.\arabic{subsection}}
\renewcommand\thesubsubsectiondis{\thesubsectiondis.\arabic{subsubsection}}


\hyphenation{op-tical net-works semi-conduc-tor}
\def\inputGnumericTable{}                                 %%

\lstset{
%language=C,
frame=single, 
breaklines=true,
columns=fullflexible
}
\begin{document}
\onecolumn

\newtheorem{theorem}{Theorem}[section]
\newtheorem{problem}{Problem}
\newtheorem{proposition}{Proposition}[section]
\newtheorem{lemma}{Lemma}[section]
\newtheorem{corollary}[theorem]{Corollary}
\newtheorem{example}{Example}[section]
\newtheorem{definition}[problem]{Definition}

\newcommand{\BEQA}{\begin{eqnarray}}
\newcommand{\EEQA}{\end{eqnarray}}
\newcommand{\define}{\stackrel{\triangle}{=}}
\bibliographystyle{IEEEtran}
\raggedbottom
\setlength{\parindent}{0pt}
\providecommand{\mbf}{\mathbf}
\providecommand{\pr}[1]{\ensuremath{\Pr\left(#1\right)}}
\providecommand{\qfunc}[1]{\ensuremath{Q\left(#1\right)}}
\providecommand{\sbrak}[1]{\ensuremath{{}\left[#1\right]}}
\providecommand{\lsbrak}[1]{\ensuremath{{}\left[#1\right.}}
\providecommand{\rsbrak}[1]{\ensuremath{{}\left.#1\right]}}
\providecommand{\brak}[1]{\ensuremath{\left(#1\right)}}
\providecommand{\lbrak}[1]{\ensuremath{\left(#1\right.}}
\providecommand{\rbrak}[1]{\ensuremath{\left.#1\right)}}
\providecommand{\cbrak}[1]{\ensuremath{\left\{#1\right\}}}
\providecommand{\lcbrak}[1]{\ensuremath{\left\{#1\right.}}
\providecommand{\rcbrak}[1]{\ensuremath{\left.#1\right\}}}
\theoremstyle{remark}
\newtheorem{rem}{Remark}
\newcommand{\sgn}{\mathop{\mathrm{sgn}}}
\providecommand{\abs}[1]{\left\vert#1\right\vert}
\providecommand{\res}[1]{\Res\displaylimits_{#1}} 
\providecommand{\norm}[1]{\left\lVert#1\right\rVert}
%\providecommand{\norm}[1]{\lVert#1\rVert}
\providecommand{\mtx}[1]{\mathbf{#1}}
\providecommand{\mean}[1]{E\left[ #1 \right]}
\providecommand{\fourier}{\overset{\mathcal{F}}{ \rightleftharpoons}}
%\providecommand{\hilbert}{\overset{\mathcal{H}}{ \rightleftharpoons}}
\providecommand{\system}{\overset{\mathcal{H}}{ \longleftrightarrow}}
	%\newcommand{\solution}[2]{\textbf{Solution:}{#1}}
\newcommand{\solution}{\noindent \textbf{Solution: }}
\newcommand{\cosec}{\,\text{cosec}\,}
\providecommand{\dec}[2]{\ensuremath{\overset{#1}{\underset{#2}{\gtrless}}}}
\newcommand{\myvec}[1]{\ensuremath{\begin{pmatrix}#1\end{pmatrix}}}
\newcommand{\mydet}[1]{\ensuremath{\begin{vmatrix}#1\end{vmatrix}}}
\numberwithin{equation}{subsection}
\makeatletter
\@addtoreset{figure}{problem}
\makeatother
\let\StandardTheFigure\thefigure
\let\vec\mathbf
\renewcommand{\thefigure}{\theproblem}
\def\putbox#1#2#3{\makebox[0in][l]{\makebox[#1][l]{}\raisebox{\baselineskip}[0in][0in]{\raisebox{#2}[0in][0in]{#3}}}}
     \def\rightbox#1{\makebox[0in][r]{#1}}
     \def\centbox#1{\makebox[0in]{#1}}
     \def\topbox#1{\raisebox{-\baselineskip}[0in][0in]{#1}}
     \def\midbox#1{\raisebox{-0.5\baselineskip}[0in][0in]{#1}}
\vspace{3cm}
\title{Assignment 12}
\author{Rubeena Aafreen}
\maketitle
\bigskip
\renewcommand{\thefigure}{\theenumi}
\renewcommand{\thetable}{\theenumi}
%
Download the latex-tikz codes from 
%
\begin{lstlisting}
https://github.com/rubeenaafreen20/EE5609/tree/master/Assignment12
\end{lstlisting}
\section{\textbf{Problem}}
(UGC-june2017,74) : \\
%
For any $n\times n$ matrix $B$, let $N(B)=\{X\in \mathbb{R}^n:BX=0\}$ be the null space of $B$. Let $A$ be a $4\times 4$ matrix with $dim(N(A-2I))=2, dim(N(A-4I))=1$ and $rank(A)=3$
Which of the following are true?\\
\begin{enumerate}
\item 0,2 and 4 are eigenvalues of A
\item determinant(A)=0
\item A is not diagonalizable
\item trace(A)=8
\end{enumerate}
\section{\textbf{Solution}}
\begin{longtable}{|c|l|}
    \hline
	\multirow{5}{*}{Given} 
	& \\
	& $A$ is a $4\times 4$ matrix.\\
	& $dim\brak{N\brak{A-2I}}=2$,\\
	& $dim\brak{N\brak{A-4I}}=1$, and\\
	& $rank\brak{A}=3$\\
	\hline
	\multirow{3}{*}{Eigenvalues of a matrix} 
	&\\
	& The number $\lambda$ is an eigenvalue of a matrix A if and only if$A-\lambda I$ is singular,\\
	& i.e. $\mydet{A-\lambda I}=0$ \\
	&\\
	& For $\lambda=2$\\
	& Given, $dim\brak{N\brak{A-2I}}=2$\\
	& $\implies nullity(A-2I)=2$\\
	& $rank(A)+nullity(A)=n$\\
	& $\implies rank\brak{A-2I}=4-2=2$\\
	& $\implies \brak{A-2I}$ is not a full rank matrix\\
	& Therefore $\mydet{A-2I}=0$\\
	&\\
	& Also,\\
	& $\implies N\brak{A-2I}=\{X\in \mathbb{R}^4:(A-2I)X=0\}$ \\
	& $\implies (A-2I)X=0$ gives two eigen vectors\\
	& $\implies$ 2 is an eigenvalue of $A$ with multiplicity 2.\\
	&\\
	& Similarly, for $\lambda=4$\\
	& Given, $dim\brak{N\brak{A-4I}}=1$\\
	& $\implies rank\brak{A-4I}=4-1=3$\\
	& $\implies \brak{A-4I}$ is not a full rank matrix\\
	\hline \newpage \hline 
	&\\
	& Therefore $\mydet{A-4I}=0$\\
	& $\implies$ 4 is an eigenvalue of $A$ with multiplicity 1.\\
	&\\
	& For $\lambda=0$\\
	& Given that $rank\brak{A}=3$\\
	& $\implies A$ is not a full rank matrix\\
	& Therefore $\mydet{A}=0$\\
	& $\implies$ 0 is an eigenvalue of $A$ with multiplicity 1.\\
	\hline
	\multirow{3}{*}{Determinant} & \\
	& Given that $rank\brak{A}=3$\\
	& $\implies A$ is not a full rank matrix\\
	& Therefore $\mydet{A}=0$\\
	&\\
	\hline
    \multirow{3}{*}{Diagonalizability} 
    & \\
    & An $n\times n$ matrix $A$ is diagonalizable if and only if $A$ has n linearly independent \\& eigen vectors.\\
	& $rank\brak{A}+nullity\brak{A}=n$\\
	& $\implies$ for $\lambda=0$,\\
	& $nullity\brak{A-\lambda I}=nullity\brak{A}=4-3=1$\\
	& $\implies$ There exists only one linearly independent eigen vector corresponding to \\& $0$ eigen value\\
	& Thus, matrix A is not diagonalizable.\\
	&\\
    \hline
    \multirow{3}{*}{Trace} & \\
	& $Trace\brak{A}$=sum of eigen values\\
	& $\implies Trace\brak{A}=0+2+2+4=8$\\
	&\\
	\hline
	\multirow{3}{*}{Conclusion} & \\
	& Option (1), (2) and (4) are correct\\
	&\\
	\hline
\end{longtable}
\end{document}