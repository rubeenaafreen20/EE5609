\documentclass[journal,12pt]{IEEEtran}
\usepackage{longtable}
\usepackage{setspace}
\usepackage{gensymb}
\singlespacing
\usepackage[cmex10]{amsmath}
\newcommand\myemptypage{
	\null
	\thispagestyle{empty}
	\addtocounter{page}{-1}
	\newpage
}
\usepackage{amsthm}
\usepackage{mdframed}
\usepackage{mathrsfs}
\usepackage{txfonts}
\usepackage{stfloats}
\usepackage{bm}
\usepackage{cite}
\usepackage{cases}
\usepackage{subfig}

\usepackage{longtable}
\usepackage{multirow}

\usepackage{enumitem}
\usepackage{mathtools}
\usepackage{steinmetz}
\usepackage{tikz}
\usepackage{circuitikz}
\usepackage{verbatim}
\usepackage{tfrupee}
\usepackage[breaklinks=true]{hyperref}
\usepackage{graphicx}
\usepackage{tkz-euclide}

\usetikzlibrary{calc,math}
\usepackage{listings}
    \usepackage{color}                                            %%
    \usepackage{array}                                            %%
    \usepackage{longtable}                                        %%
    \usepackage{calc}                                             %%
    \usepackage{multirow}                                         %%
    \usepackage{hhline}                                           %%
    \usepackage{ifthen}                                           %%
    \usepackage{lscape}     
\usepackage{multicol}
\usepackage{chngcntr}

\DeclareMathOperator*{\Res}{Res}

\renewcommand\thesection{\arabic{section}}
\renewcommand\thesubsection{\thesection.\arabic{subsection}}
\renewcommand\thesubsubsection{\thesubsection.\arabic{subsubsection}}

\renewcommand\thesectiondis{\arabic{section}}
\renewcommand\thesubsectiondis{\thesectiondis.\arabic{subsection}}
\renewcommand\thesubsubsectiondis{\thesubsectiondis.\arabic{subsubsection}}


\hyphenation{op-tical net-works semi-conduc-tor}
\def\inputGnumericTable{}                                 %%

\lstset{
%language=C,
frame=single, 
breaklines=true,
columns=fullflexible
}
\begin{document}
\onecolumn

\newtheorem{theorem}{Theorem}[section]
\newtheorem{problem}{Problem}
\newtheorem{proposition}{Proposition}[section]
\newtheorem{lemma}{Lemma}[section]
\newtheorem{corollary}[theorem]{Corollary}
\newtheorem{example}{Example}[section]
\newtheorem{definition}[problem]{Definition}

\newcommand{\BEQA}{\begin{eqnarray}}
\newcommand{\EEQA}{\end{eqnarray}}
\newcommand{\define}{\stackrel{\triangle}{=}}
\bibliographystyle{IEEEtran}
\raggedbottom
\setlength{\parindent}{0pt}
\providecommand{\mbf}{\mathbf}
\providecommand{\pr}[1]{\ensuremath{\Pr\left(#1\right)}}
\providecommand{\qfunc}[1]{\ensuremath{Q\left(#1\right)}}
\providecommand{\sbrak}[1]{\ensuremath{{}\left[#1\right]}}
\providecommand{\lsbrak}[1]{\ensuremath{{}\left[#1\right.}}
\providecommand{\rsbrak}[1]{\ensuremath{{}\left.#1\right]}}
\providecommand{\brak}[1]{\ensuremath{\left(#1\right)}}
\providecommand{\lbrak}[1]{\ensuremath{\left(#1\right.}}
\providecommand{\rbrak}[1]{\ensuremath{\left.#1\right)}}
\providecommand{\cbrak}[1]{\ensuremath{\left\{#1\right\}}}
\providecommand{\lcbrak}[1]{\ensuremath{\left\{#1\right.}}
\providecommand{\rcbrak}[1]{\ensuremath{\left.#1\right\}}}
\theoremstyle{remark}
\newtheorem{rem}{Remark}
\newcommand{\sgn}{\mathop{\mathrm{sgn}}}
\providecommand{\abs}[1]{\left\vert#1\right\vert}
\providecommand{\res}[1]{\Res\displaylimits_{#1}} 
\providecommand{\norm}[1]{\left\lVert#1\right\rVert}
%\providecommand{\norm}[1]{\lVert#1\rVert}
\providecommand{\mtx}[1]{\mathbf{#1}}
\providecommand{\mean}[1]{E\left[ #1 \right]}
\providecommand{\fourier}{\overset{\mathcal{F}}{ \rightleftharpoons}}
%\providecommand{\hilbert}{\overset{\mathcal{H}}{ \rightleftharpoons}}
\providecommand{\system}{\overset{\mathcal{H}}{ \longleftrightarrow}}
	%\newcommand{\solution}[2]{\textbf{Solution:}{#1}}
\newcommand{\solution}{\noindent \textbf{Solution: }}
\newcommand{\cosec}{\,\text{cosec}\,}
\providecommand{\dec}[2]{\ensuremath{\overset{#1}{\underset{#2}{\gtrless}}}}
\newcommand{\myvec}[1]{\ensuremath{\begin{pmatrix}#1\end{pmatrix}}}
\newcommand{\mydet}[1]{\ensuremath{\begin{vmatrix}#1\end{vmatrix}}}
\numberwithin{equation}{subsection}
\makeatletter
\@addtoreset{figure}{problem}
\makeatother
\let\StandardTheFigure\thefigure
\let\vec\mathbf
\renewcommand{\thefigure}{\theproblem}
\def\putbox#1#2#3{\makebox[0in][l]{\makebox[#1][l]{}\raisebox{\baselineskip}[0in][0in]{\raisebox{#2}[0in][0in]{#3}}}}
     \def\rightbox#1{\makebox[0in][r]{#1}}
     \def\centbox#1{\makebox[0in]{#1}}
     \def\topbox#1{\raisebox{-\baselineskip}[0in][0in]{#1}}
     \def\midbox#1{\raisebox{-0.5\baselineskip}[0in][0in]{#1}}
\vspace{3cm}
\title{Assignment 13}
\author{Rubeena Aafreen}
\maketitle
\bigskip
\renewcommand{\thefigure}{\theenumi}
\renewcommand{\thetable}{\theenumi}
%
Download the latex-tikz codes from 
%
\begin{lstlisting}
https://github.com/rubeenaafreen20/EE5609/tree/master/Assignment13
\end{lstlisting}
\section{\textbf{Problem}}
%
Suppose that $\vec{A}$ is a $2\times2$ matrix with real entries which is symmetric $\brak{\vec{A}^t=\vec{A}}$.
Prove that $\vec{A}$ is similar over $\mathbb{R}$ to a diagonal matrix. 
\section{\textbf{Solution}}
\begin{longtable}{|c|l|}
    \hline
	\multirow{5}{*}{Given} 
	& \\
	& $\vec{A}$ is a $2\times2$ matrix with real entries\\
	& and $\vec{A}$ is symmetric $\brak{\vec{A}^t=\vec{A}}$\\
	&\\
	\hline
	\multirow{3}{*}{To Prove} 
	&\\
	& $\vec{A}$ is similar to diagonal matrix over $\mathbb{R}$\\
	&\\
	\hline
	\multirow{3}{*}{Theory} & \\
	& $\vec{A}$ is similar to diagonal matrix $\vec{\Lambda}$ if\\
	&$\exists$ an invertible matrix $\vec{P}$ such that:\\
	& $\vec{A}=\vec{P}\vec{\Lambda} \vec{P}^{-1}$\\
	&\\
	\hline
	\multirow{3}{*}{Proof} & \\
	& Let $\vec{A}=\myvec{a&c\\c&b}, a,b,c \in \mathbb{R}$\\
	&\\
	& Characteristic polynomial:\\
	& $p(t)=\mydet{\vec{A}-\lambda\vec{I}}$\\
	& $p\brak{t}=\mydet{a-t&c\\c&b-t}$\\
	&\\
	& $\implies p\brak{t}=t^2-(a+b)t+ab-c^2=0$\\
	& Roots of p(t) are eigenvalues of $\vec{A}$\\
	&\\
	& Discriminant of $p\brak{t}$ is given by\\
	& $(a+b)^2-4(ab-c^2)=a^2+b^2-2ab+c^2$\\
	& $\qquad\qquad\qquad\qquad=(a-b)^2+4c^2>0$\\
	&\\
	& We observe that the above equation has positive discriminant, hence $\lambda$ \\
	& has real values\\
	&\\
	\hline \newpage \hline
	&\\
	& Eigen vectors are obtained by:\\
	& $\brak{\vec{A}-\lambda \vec{I}}\vec{X}=0$\\
	& Since, eigen values are real\\
	& $\implies$ eigen vectors $\vec{v_1}$ and $\vec{v_2}$ are linearly independent.\\
	&\\
	& Let $\vec{P}=\myvec{\vec{v_1}& \vec{v_2}}$\\
	& $\implies \vec{A}\vec{P}=\myvec{\lambda_1 \vec{v_1}&\lambda_2 \vec{v_2}}$\\
	& $\implies \vec{A}\vec{P}=\vec{P}\vec{\Lambda}$, \\
	& where $\vec{\Lambda}=\myvec{\lambda_1&0\\0&\lambda_2}$\\
	& $\implies \vec{A}=\vec{P}\vec{\Lambda} \vec{P}^{-1}$\\
	&\\
	& Therefore, $\vec{A}$ is similar to diagonal matrix $\vec{\Lambda}$\\
	& Hence, Proved.
	\hline
\end{longtable}
\end{document}