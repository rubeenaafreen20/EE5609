\documentclass[journal,12pt,twocolumn]{IEEEtran}

\usepackage{setspace}
\usepackage{gensymb}

\singlespacing


\usepackage[cmex10]{amsmath}

\usepackage{amsthm}

\usepackage{mathrsfs}
\usepackage{txfonts}
\usepackage{stfloats}
\usepackage{bm}
\usepackage{cite}
\usepackage{cases}
\usepackage{subfig}

\usepackage{longtable}
\usepackage{multirow}

\usepackage{enumitem}
\usepackage{mathtools}
\usepackage{steinmetz}
\usepackage{tikz}
\usepackage{circuitikz}
\usepackage{verbatim}
\usepackage{tfrupee}
\usepackage[breaklinks=true]{hyperref}

\usepackage{tkz-euclide}

\usetikzlibrary{calc,math}
\usepackage{listings}
    \usepackage{color}                                            %%
    \usepackage{array}                                            %%
    \usepackage{longtable}                                        %%
    \usepackage{calc}                                             %%
    \usepackage{multirow}                                         %%
    \usepackage{hhline}                                           %%
    \usepackage{ifthen}                                           %%
    \usepackage{lscape}     
\usepackage{multicol}
\usepackage{chngcntr}

\DeclareMathOperator*{\Res}{Res}

\renewcommand\thesection{\arabic{section}}
\renewcommand\thesubsection{\thesection.\arabic{subsection}}
\renewcommand\thesubsubsection{\thesubsection.\arabic{subsubsection}}

\renewcommand\thesectiondis{\arabic{section}}
\renewcommand\thesubsectiondis{\thesectiondis.\arabic{subsection}}
\renewcommand\thesubsubsectiondis{\thesubsectiondis.\arabic{subsubsection}}


\hyphenation{op-tical net-works semi-conduc-tor}
\def\inputGnumericTable{}                                 %%

\lstset{
%language=C,
frame=single, 
breaklines=true,
columns=fullflexible
}
\begin{document}


\newtheorem{theorem}{Theorem}[section]
\newtheorem{problem}{Problem}
\newtheorem{proposition}{Proposition}[section]
\newtheorem{lemma}{Lemma}[section]
\newtheorem{corollary}[theorem]{Corollary}
\newtheorem{example}{Example}[section]
\newtheorem{definition}[problem]{Definition}

\newcommand{\BEQA}{\begin{eqnarray}}
\newcommand{\EEQA}{\end{eqnarray}}
\newcommand{\define}{\stackrel{\triangle}{=}}
\bibliographystyle{IEEEtran}
\providecommand{\mbf}{\mathbf}
\providecommand{\pr}[1]{\ensuremath{\Pr\left(#1\right)}}
\providecommand{\qfunc}[1]{\ensuremath{Q\left(#1\right)}}
\providecommand{\sbrak}[1]{\ensuremath{{}\left[#1\right]}}
\providecommand{\lsbrak}[1]{\ensuremath{{}\left[#1\right.}}
\providecommand{\rsbrak}[1]{\ensuremath{{}\left.#1\right]}}
\providecommand{\brak}[1]{\ensuremath{\left(#1\right)}}
\providecommand{\lbrak}[1]{\ensuremath{\left(#1\right.}}
\providecommand{\rbrak}[1]{\ensuremath{\left.#1\right)}}
\providecommand{\cbrak}[1]{\ensuremath{\left\{#1\right\}}}
\providecommand{\lcbrak}[1]{\ensuremath{\left\{#1\right.}}
\providecommand{\rcbrak}[1]{\ensuremath{\left.#1\right\}}}
\theoremstyle{remark}
\newtheorem{rem}{Remark}
\newcommand{\sgn}{\mathop{\mathrm{sgn}}}
\providecommand{\abs}[1]{\left\vert#1\right\vert}
\providecommand{\res}[1]{\Res\displaylimits_{#1}} 
\providecommand{\norm}[1]{\left\lVert#1\right\rVert}
%\providecommand{\norm}[1]{\lVert#1\rVert}
\providecommand{\mtx}[1]{\mathbf{#1}}
\providecommand{\mean}[1]{E\left[ #1 \right]}
\providecommand{\fourier}{\overset{\mathcal{F}}{ \rightleftharpoons}}
%\providecommand{\hilbert}{\overset{\mathcal{H}}{ \rightleftharpoons}}
\providecommand{\system}{\overset{\mathcal{H}}{ \longleftrightarrow}}
	%\newcommand{\solution}[2]{\textbf{Solution:}{#1}}
\newcommand{\solution}{\noindent \textbf{Solution: }}
\newcommand{\cosec}{\,\text{cosec}\,}
\providecommand{\dec}[2]{\ensuremath{\overset{#1}{\underset{#2}{\gtrless}}}}
\newcommand{\myvec}[1]{\ensuremath{\begin{pmatrix}#1\end{pmatrix}}}
\newcommand{\mydet}[1]{\ensuremath{\begin{vmatrix}#1\end{vmatrix}}}
\numberwithin{equation}{subsection}
\makeatletter
\@addtoreset{figure}{problem}
\makeatother
\let\StandardTheFigure\thefigure
\let\vec\mathbf
\renewcommand{\thefigure}{\theproblem}
\def\putbox#1#2#3{\makebox[0in][l]{\makebox[#1][l]{}\raisebox{\baselineskip}[0in][0in]{\raisebox{#2}[0in][0in]{#3}}}}
     \def\rightbox#1{\makebox[0in][r]{#1}}
     \def\centbox#1{\makebox[0in]{#1}}
     \def\topbox#1{\raisebox{-\baselineskip}[0in][0in]{#1}}
     \def\midbox#1{\raisebox{-0.5\baselineskip}[0in][0in]{#1}}
\vspace{3cm}
\title{Assignment 8}
\author{Rubeena Aafreen}
\maketitle
\newpage
\bigskip
\renewcommand{\thefigure}{\theenumi}
\renewcommand{\thetable}{\theenumi}
Download all solutions from 
\begin{lstlisting}
https://github.com/rubeenaafreen20/EE5609/tree/master/Assignment8
\end{lstlisting}
%
%
\section{Problem}
Let \textit{s\textless n} and A an $s\times n$ matrix with entries in the field $\mathbb{F}$. Use Theorem 4 to show that there is a non-zero $\vec{x}$ in $\mathbb{F}^{n\times 1}$ such that $Ax=\vec{0}$. 
\section{Explanation}
\textbf{Theorem 4:}\textit{Let $\mathbb{V}$ be a vector space which is spanned by a finite set of vectors $\beta_1,\beta_2,...,\beta_m$. Then any independent set of vectors in $\mathbb{V}$ is finite and contains no more than m elements.}
\section{Solution}

Let \mathbb{V} be a vector space spanned by $A_1,A_2,\dots,A_n$, where $A_i$, i=1,2,\dots,n are columns of matrix $A_{s\times n}$. let $\vec{x}$ be a vector in vector space $\mathbb{V}$\newline
Let
    \begin{multiline}
    \begin{align}
    S=\{x:Ax=\vec{0}\}\\
    \text{If } x_1,x_2 \in S\\
    \implies A(x_1+x_2)=Ax_1+Ax_2=0\\
    \implies (x_1+x_2)\in S \label{eq1subspace}\\
    \text{If } x \in S\\
    \implies A(\alpha x)=\alpha(Ax)=0\\
    \implies \alpha x\in S \label{eq2subspace}
    \end{align}
    \end{multiline}
From equations \eqref{eq1subspace} and \eqref{eq2subspace}, $S$ is a subspace of $\mathbb{V}$.\newline
From equation \eqref{eq1subspace}, it is clear that,
\begin{align}
    \vec{x}=\myvec{x_1\\x_2\\\vdots\\x_n}
\end{align}
 is in $S$ iff,
 \begin{align}
    \sum_{i=1}^{n}A_ix_i = \vec{0}
 \end{align}
 Where $A_i$ is the $i^{th}$ column of matrix A.\newline
 Let $A_1,A_2,\dots,A_s$ form a basis for column space of $A$, then,
 \begin{align}
    &Ax=\vec{0}\\
    \implies &A_1x_1+\dots+A_sx_s+A_{s+1}x_{s+1}+\dots+A_nx_n=\vec{0}\label{eqhomogeneous}
 \end{align}
The columns not in the basis can be expressed as linear combinations of the basis columns
\begin{align}
    A_{s+1}=\sum_{j=1}^{s}\beta_{1,j}x_i\\
    A_{s+2}=\sum_{j=1}^{s}\beta_{2,j}x_i\\
    \vdots\\
    A_{n}=\sum_{j=1}^{s}\beta_{(n-s),j}x_i
\end{align}
Therefore, we can conclude that the columns of matrix A are not independent.\newline
Equation \eqref{eqhomogeneous} leads to a homogeneous system of linear equations with \textit{s equations and n unknowns}.\\
Since, $s$ rows can hold at most $s$ pivots, there must be $(n-s)$ free variables.\newline
These free variables can be assigned any value. Hence, there are more solutions to equation \eqref{eqhomogeneous} than the trivial $x=0$.\newline
$\implies$ Equation \eqref{eqhomogeneous} will have at least one special solution. \newline
Therefore, at least one $x_j\neq0, j=1,2,\dots,n$\ exists such that $Ax=\vec{0}$.
\end{document}
