\documentclass[journal,12pt,twocolumn]{IEEEtran}
%
\usepackage{setspace}
\usepackage{gensymb}
\usepackage{siunitx}
\usepackage{tkz-euclide} 
\usepackage{textcomp}
\usepackage{standalone}
\usetikzlibrary{calc}
\newcommand\hmmax{0}
\newcommand\bmmax{0}

%\doublespacing
\singlespacing

%\usepackage{graphicx}
%\usepackage{amssymb}
%\usepackage{relsize}
\usepackage[cmex10]{amsmath}
%\usepackage{amsthm}
%\interdisplaylinepenalty=2500
%\savesymbol{iint}
%\usepackage{txfonts}
%\restoresymbol{TXF}{iint}
%\usepackage{wasysym}
\usepackage{amsthm}
%\usepackage{iithtlc}
\usepackage{mathrsfs}
\usepackage{txfonts}
\usepackage{stfloats}
\usepackage{bm}
\usepackage{cite}
\usepackage{cases}
\usepackage{subfig}
%\usepackage{xtab}
\usepackage{longtable}
\usepackage{multirow}
%\usepackage{algorithm}
%\usepackage{algpseudocode}
\usepackage{enumitem}
\usepackage{mathtools}
\usepackage{steinmetz}
\usepackage{tikz}
\usepackage{circuitikz}
\usepackage{verbatim}
\usepackage{tfrupee}
\usepackage[breaklinks=true]{hyperref}
%\usepackage{stmaryrd}
\usepackage{tkz-euclide} % loads  TikZ and tkz-base
%\usetkzobj{all}
\usetikzlibrary{calc,math}
\usepackage{listings}
    \usepackage{color}                                            %%
    \usepackage{array}                                            %%
    \usepackage{longtable}                                        %%
    \usepackage{calc}                                             %%
    \usepackage{multirow}                                         %%
    \usepackage{hhline}                                           %%
    \usepackage{ifthen}                                           %%
  %optionally (for landscape tables embedded in another document): %%
    \usepackage{lscape}     
\usepackage{multicol}
\usepackage{chngcntr}
\usepackage{amsmath}
\usepackage{cleveref}
%\usepackage{enumerate}

%\usepackage{wasysym}
%\newcounter{MYtempeqncnt}
\DeclareMathOperator*{\Res}{Res}
%\renewcommand{\baselinestretch}{2}
\renewcommand\thesection{\arabic{section}}
\renewcommand\thesubsection{\thesection.\arabic{subsection}}
\renewcommand\thesubsubsection{\thesubsection.\arabic{subsubsection}}

\renewcommand\thesectiondis{\arabic{section}}
\renewcommand\thesubsectiondis{\thesectiondis.\arabic{subsection}}
\renewcommand\thesubsubsectiondis{\thesubsectiondis.\arabic{subsubsection}}

% correct bad hyphenation here
\hyphenation{op-tical net-works semi-conduc-tor}
\def\inputGnumericTable{}                                 %%

\lstset{
%language=C,
frame=single, 
breaklines=true,
columns=fullflexible
}
%\lstset{
%language=tex,
%frame=single, 
%breaklines=true
%}
\usepackage{graphicx}
\usepackage{pgfplots}

\begin{document}


\newtheorem{theorem}{Theorem}[section]
\newtheorem{problem}{Problem}
\newtheorem{proposition}{Proposition}[section]
\newtheorem{lemma}{Lemma}[section]
\newtheorem{corollary}[theorem]{Corollary}
\newtheorem{example}{Example}[section]
\newtheorem{definition}[problem]{Definition}
%\newtheorem{thm}{Theorem}[section] 
%\newtheorem{defn}[thm]{Definition}
%\newtheorem{algorithm}{Algorithm}[section]
%\newtheorem{cor}{Corollary}
\newcommand{\BEQA}{\begin{eqnarray}}
\newcommand{\EEQA}{\end{eqnarray}}
\newcommand{\define}{\stackrel{\triangle}{=}}
\bibliographystyle{IEEEtran}
%\bibliographystyle{ieeetr}
\providecommand{\mbf}{\mathbf}
\providecommand{\abs}[1]{\ensuremath{\left\vert#1\right\vert}}
\providecommand{\norm}[1]{\ensuremath{\left\lVert#1\right\rVert}}
\providecommand{\mean}[1]{\ensuremath{E\left[ #1 \right]}}
\providecommand{\pr}[1]{\ensuremath{\Pr\left(#1\right)}}
\providecommand{\qfunc}[1]{\ensuremath{Q\left(#1\right)}}
\providecommand{\sbrak}[1]{\ensuremath{{}\left[#1\right]}}
\providecommand{\lsbrak}[1]{\ensuremath{{}\left[#1\right.}}
\providecommand{\rsbrak}[1]{\ensuremath{{}\left.#1\right]}}
\providecommand{\brak}[1]{\ensuremath{\left(#1\right)}}
\providecommand{\lbrak}[1]{\ensuremath{\left(#1\right.}}
\providecommand{\rbrak}[1]{\ensuremath{\left.#1\right)}}
\providecommand{\cbrak}[1]{\ensuremath{\left\{#1\right\}}}
\providecommand{\lcbrak}[1]{\ensuremath{\left\{#1\right.}}
\providecommand{\rcbrak}[1]{\ensuremath{\left.#1\right\}}}
\theoremstyle{remark}
\newtheorem{rem}{Remark}
\newcommand{\sgn}{\mathop{\mathrm{sgn}}}
\providecommand{\res}[1]{\Res\displaylimits_{#1}} 
%\providecommand{\norm}[1]{\lVert#1\rVert}
\providecommand{\mtx}[1]{\mathbf{#1}}
\providecommand{\fourier}{\overset{\mathcal{F}}{ \rightleftharpoons}}
%\providecommand{\hilbert}{\overset{\mathcal{H}}{ \rightleftharpoons}}
\providecommand{\system}{\overset{\mathcal{H}}{ \longleftrightarrow}}
	%\newcommand{\solution}[2]{\textbf{Solution:}{#1}}
\newcommand{\solution}{\noindent \textbf{Solution: }}
\newcommand{\cosec}{\,\text{cosec}\,}
\providecommand{\dec}[2]{\ensuremath{\overset{#1}{\underset{#2}{\gtrless}}}}
\newcommand{\myvec}[1]{\ensuremath{\begin{pmatrix}#1\end{pmatrix}}}
\newcommand{\mydet}[1]{\ensuremath{\begin{vmatrix}#1\end{vmatrix}}}
%\numberwithin{equation}{section}
\numberwithin{equation}{subsection}
%\numberwithin{problem}{section}
%\numberwithin{definition}{section}
\makeatletter
\@addtoreset{figure}{problem}
\makeatother
\let\StandardTheFigure\thefigure
\let\vec\mathbf
%\renewcommand{\thefigure}{\theproblem.\arabic{figure}}
\renewcommand{\thefigure}{\theproblem}
%\setlist[enumerate,1]{before=\renewcommand\theequation{\theenumi.\arabic{equation}}
%\counterwithin{equation}{enumi}
%\renewcommand{\theequation}{\arabic{subsection}.\arabic{equation}}
\def\putbox#1#2#3{\makebox[0in][l]{\makebox[#1][l]{}\raisebox{\baselineskip}[0in][0in]{\raisebox{#2}[0in][0in]{#3}}}}
     \def\rightbox#1{\makebox[0in][r]{#1}}
     \def\centbox#1{\makebox[0in]{#1}}
     \def\topbox#1{\raisebox{-\baselineskip}[0in][0in]{#1}}
     
 \vspace{3cm}
 \title{Assignment 5}
 \author{Rubeena Aafreen}
 \maketitle
 \newpage
 \bigskip
 %\renewcommand{\thefigure}{\theenumi}
 \renewcommand{\thetable}{\theenumi}
\vspace{1.0cm}
\vspace{2ex}
\begin{center}
{\underline{\Large \bf QR Decomposition}}
\end{center}
\begin{abstract}
This document solves problem based on QR decomposition.
\end{abstract}
\vspace{0.5cm}
%
Download all solutions from 
\begin{lstlisting}
https://github.com/rubeenaafreen20/EE5609/tree/master/Assignment5
\end{lstlisting}
%
%
\vspace{0.5mm}
\section{Problem}
Perform QR decomposition on the matrix $\vec{A}$ 
\begin{align}
\vec{A}=\myvec{1&3\\2&4} \label{eq:1}
\end{align}
\section{Solution}
The columns of the matrix $\vec{A}$ can be represented as:
\begin{align}
\alpha=\myvec{1\\2}\label{eq:2}\\
\beta=\myvec{3\\4}\label{eq:3}
\end{align}
For QR decomposition, matrix A is represented in the form:
\begin{align}
    \Vec{A}=\vec{Q}\vec{R}
\end{align}
where $\vec{Q}$ and $\Vec{R}$ are:
\begin{align}
    \vec{Q}=\myvec{\Vec{u_1} & \vec{u_2}}\label{eq:4}\\
    \vec{R}=\myvec{k_1 & r_1 \\ 0 & k_2}\label{eq:5}
\end{align}
Here $\vec{R}$ is a upper triangular matrix and Q is a orthogonal matrix such that,
\begin{align}
\vec{Q^T}\vec{Q}=\vec{I}
\end{align}
Now we calculate the above values,
\begin{align}
k_1 = \norm{\alpha}\\
\vec{u_1}=\frac{\alpha}{k_1}\\
r_1 = \frac{\vec{u_1^T}\beta}{\norm{\Vec{u_1}}^2}\\
\vec{u_2}=\frac{\beta-r_1\vec{u_1}}{\norm{\beta-r_1\vec{u_1}}}\\
k_2 = \vec{u_2^T}\beta
\end{align}
Substituting \eqref{eq:2} and \eqref{eq:3} in the above equations, we get
\begin{align}
k_1= \sqrt{1^2+2^2}= \sqrt{5}\\
\vec{u_1}= \frac{1}{\sqrt{5}}\myvec{1\\2}=\myvec{\frac{1}{\sqrt{5}}\\\frac{2}{\sqrt{5}}}\\
r_1=\frac{1}{\brak{\sqrt{\frac{1}{5}+\frac{4}{5}}}^2}\myvec{\frac{1}{\sqrt{5}}&\frac{2}{\sqrt{5}}}\myvec{3\\4}\\
\implies r_1= \frac{11}{\sqrt{5}}\\
\beta-r_1\vec{u_1}=\myvec{3\\4}-\myvec{\frac{11}{5}\\\frac{22}{5}}=\myvec{\frac{4}{5}\\\frac{-2}{5}}\\
\Vec{u_2}=\frac{\myvec{\frac{4}{5}\\\frac{-2}{5}}}{\sqrt{{\frac{4}{5}}^2+{\frac{-2}{5}}^2}}=\myvec{\frac{2}{\sqrt{5}}\\\frac{-1}{\sqrt{5}}}\\
k_2 = \myvec{\frac{2}{\sqrt{5}}&\frac{-1}{\sqrt{5}}}\myvec{3\\4}= \frac{2}{\sqrt{5}}
\end{align}
Therefore, from \eqref{eq:4} and \eqref{eq:5} we get,
\begin{align}
\vec{Q}= \myvec{\frac{1}{\sqrt{5}}&\frac{2}{\sqrt{5}}\\\frac{2}{\sqrt{5}}&\frac{-1}{\sqrt{5}}}\\
\Vec{R}=\myvec{\sqrt{5}&\frac{11}{\sqrt{5}}\\0&\frac{2}{\sqrt{5}}}
\end{align}
where
\begin{align}
    \vec{Q}^T\vec{Q}=\myvec{\frac{1}{\sqrt{5}}&\frac{2}{\sqrt{5}}\\\frac{2}{\sqrt{5}}&\frac{-1}{\sqrt{5}}}\myvec{\frac{1}{\sqrt{5}}&\frac{2}{\sqrt{5}}\\\frac{2}{\sqrt{5}}&\frac{-1}{\sqrt{5}}}    =\myvec{1&0\\0&1}=\Vec{I}
\end{align}
Therefore matrix $\vec{A}$ in QR decomposed form is,
\begin{align}
\myvec{1&3\\2&4}=\myvec{\frac{1}{\sqrt{5}}&\frac{2}{\sqrt{5}}\\\frac{2}{\sqrt{5}}&\frac{-1}{\sqrt{5}}}\myvec{\sqrt{5}&\frac{11}{\sqrt{5}}\\0&\frac{2}{\sqrt{5}}}
\end{align}
\end{document}