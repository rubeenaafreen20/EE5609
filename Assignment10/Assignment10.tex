\documentclass[journal,12pt,twocolumn]{IEEEtran}

\usepackage{setspace}
\usepackage{gensymb}

\singlespacing


\usepackage[cmex10]{amsmath}
%\usepackage{amsthm}
%\interdisplaylinepenalty=2500
%\savesymbol{iint}
%\usepackage{txfonts}
%\restoresymbol{TXF}{iint}
%\usepackage{wasysym}
\usepackage{amsthm}
%\usepackage{iithtlc}
\usepackage{mathrsfs}
\usepackage{txfonts}
\usepackage{stfloats}
\usepackage{bm}
\usepackage{cite}
\usepackage{cases}
\usepackage{subfig}
%\usepackage{xtab}
\usepackage{longtable}
\usepackage{multirow}
%\usepackage{algorithm}
%\usepackage{algpseudocode}
\usepackage{enumitem}
\usepackage{mathtools}
\usepackage{graphicx}
\usepackage{refstyle}
\usepackage{caption}
\usepackage{steinmetz}
\usepackage{tikz}
%\usepackage{circuitikz}
\usepackage{verbatim}
\usepackage{tfrupee}
\usepackage[breaklinks=true]{hyperref}
%\usepackage{stmaryrd}
\usepackage{tkz-euclide} % loads  TikZ and tkz-base
%\usetkzobj{all}
\usetikzlibrary{calc,math}
\usepackage{listings}
   \usepackage{color}                                            %%
    \usepackage{array}                                            %%
    \usepackage{longtable}                                        %%
    \usepackage{calc}                                             %%
    \usepackage{multirow}                                         %%
    \usepackage{hhline}                                           %%
    \usepackage{ifthen}                                           %%
  %optionally (for landscape tables embedded in another document): %%
    \usepackage{lscape}     
%\usepackage{multicol}
\usepackage{chngcntr}
%\usepackage{enumerate}

%\usepackage{wasysym}
%\newcounter{MYtempeqncnt}
\DeclareMathOperator*{\Res}{Res}
%\renewcommand{\baselinestretch}{2}
\renewcommand\thesection{\arabic{section}}
\renewcommand\thesubsection{\thesection.\arabic{subsection}}
\renewcommand\thesubsubsection{\thesubsection.\arabic{subsubsection}}

\renewcommand\thesectiondis{\arabic{section}}
\renewcommand\thesubsectiondis{\thesectiondis.\arabic{subsection}}
\renewcommand\thesubsubsectiondis{\thesubsectiondis.\arabic{subsubsection}}

% correct bad hyphenation here
\hyphenation{op-tical net-works semi-conduc-tor}
\def\inputGnumericTable{}                                 %%

\lstset{
%language=C,
frame=single, 
breaklines=true,
columns=fullflexible
}

\begin{document}

\newtheorem{theorem}{Theorem}[section]
\newtheorem{problem}{Problem}
\newtheorem{proposition}{Proposition}[section]
\newtheorem{lemma}{Lemma}[section]
\newtheorem{corollary}[theorem]{Corollary}
\newtheorem{example}{Example}[section]
\newtheorem{definition}[problem]{Definition}

\newcommand{\BEQA}{\begin{eqnarray}}
\newcommand{\EEQA}{\end{eqnarray}}
\newcommand{\define}{\stackrel{\triangle}{=}}
\bibliographystyle{IEEEtran}
%\bibliographystyle{ieeetr}
\providecommand{\mbf}{\mathbf}
\providecommand{\pr}[1]{\ensuremath{\Pr\left(#1\right)}}
\providecommand{\qfunc}[1]{\ensuremath{Q\left(#1\right)}}
\providecommand{\sbrak}[1]{\ensuremath{{}\left[#1\right]}}
\providecommand{\lsbrak}[1]{\ensuremath{{}\left[#1\right.}}
\providecommand{\rsbrak}[1]{\ensuremath{{}\left.#1\right]}}
\providecommand{\brak}[1]{\ensuremath{\left(#1\right)}}
\providecommand{\lbrak}[1]{\ensuremath{\left(#1\right.}}
\providecommand{\rbrak}[1]{\ensuremath{\left.#1\right)}}
\providecommand{\cbrak}[1]{\ensuremath{\left\{#1\right\}}}
\providecommand{\lcbrak}[1]{\ensuremath{\left\{#1\right.}}
\providecommand{\rcbrak}[1]{\ensuremath{\left.#1\right\}}}
\theoremstyle{remark}
\newtheorem{rem}{Remark}
\newcommand{\sgn}{\mathop{\mathrm{sgn}}}
%\providecommand{\abs}[1]{\left\vert#1\right\vert}
\providecommand{\res}[1]{\Res\displaylimits_{#1}} 
%\providecommand{\norm}[1]{\left\lVert#1\right\rVert}
\providecommand{\norm}[1]{\lVert#1\rVert}
\providecommand{\mtx}[1]{\mathbf{#1}}
%\providecommand{\mean}[1]{E\left[ #1 \right]}
\providecommand{\fourier}{\overset{\mathcal{F}}{ \rightleftharpoons}}
%\providecommand{\hilbert}{\overset{\mathcal{H}}{ \rightleftharpoons}}
\providecommand{\system}{\overset{\mathcal{H}}{ \longleftrightarrow}}
	%\newcommand{\solution}[2]{\textbf{Solution:}{#1}}
\newcommand{\solution}{\noindent \textbf{Solution: }}
\newcommand{\cosec}{\,\text{cosec}\,}
\providecommand{\dec}[2]{\ensuremath{\overset{#1}{\underset{#2}{\gtrless}}}}
\newcommand{\myvec}[1]{\ensuremath{\begin{pmatrix}#1\end{pmatrix}}}
\newcommand{\mydet}[1]{\ensuremath{\begin{vmatrix}#1\end{vmatrix}}}
%\numberwithin{equation}{section}
\numberwithin{equation}{subsection}
%\numberwithin{problem}{section}
%\numberwithin{definition}{section}
\makeatletter
\@addtoreset{figure}{problem}
\makeatother
\let\StandardTheFigure\thefigure
\let\vec\mathbf
%\renewcommand{\thefigure}{\theproblem.\arabic{figure}}
\renewcommand{\thefigure}{\theproblem}
%\setlist[enumerate,1]{before=\renewcommand\theequation{\theenumi.\arabic{equation}}
%\counterwithin{equation}{enumi}
%\renewcommand{\theequation}{\arabic{subsection}.\arabic{equation}}
\def\putbox#1#2#3{\makebox[0in][l]{\makebox[#1][l]{}\raisebox{\baselineskip}[0in][0in]{\raisebox{#2}[0in][0in]{#3}}}}
     \def\rightbox#1{\makebox[0in][r]{#1}}
     \def\centbox#1{\makebox[0in]{#1}}
     \def\topbox#1{\raisebox{-\baselineskip}[0in][0in]{#1}}
     \def\midbox#1{\raisebox{-0.5\baselineskip}[0in][0in]{#1}}
\vspace{3cm}
\title{Assignment 10}
\author{Rubeena Aafreen}
\maketitle
\newpage
\bigskip
\renewcommand{\thefigure}{\theenumi}
\renewcommand{\thetable}{\theenumi}
\begin{abstract}
This assignment deals with annihilator of vector space.
\end{abstract}
Download  all solutions from 
\begin{lstlisting}
https://github.com/rubeenaafreen20/EE5609/tree/master/Assignment10
\end{lstlisting}
\section{Problem}
Let $\vec{W}$ be the subspace of $\mathbb{R}^5$ which is spanned by the vectors  
   \begin{multline}
    \begin{aligned}
    &\alpha_1=\epsilon_1+2\epsilon_2+\epsilon_3,\\ &\alpha_2=\epsilon_2+3\epsilon_3+3\epsilon_4+\epsilon_5,\\&\alpha_3=\epsilon_1+4\epsilon_2+6\epsilon_3+4\epsilon_4+\epsilon_5
    \end{aligned}
    \end{multline}
    Find a basis for $\vec{W^0}$
\section{Annihilator}
\subsection{Definition}
 If $\vec{V}$ is a vector space over the field $\mathbb{F}$ and $\vec{W}$ is a subset of $\vec{V}$, the annihilator of $\vec{W}$ is the set $\vec{W^0}$ of linear functionals $\vec{f}$ on $\vec{V}$ such that $\vec{f}(\alpha)=0$ for every $\alpha$ in $\vec{W}$.
\subsection{Properties of Annihilator}
If $f$ is a linear functional on $R^n$:
\begin{align}
    f\brak{x_1,x_2,\dots,x_n}=\sum_{j=1}^{n}c_j x_j\eqlabel{eqDefn}
\end{align}
Then $f$ is in $W^0$ if and only if 
\begin{align}
    \forall \alpha \in W\colon f\brak{\alpha}=0 \iff f=0
\end{align}
\section{Solution}
\eqref{eqDefn} can be expressed as:
\begin{align}
    \vec{f(x)}=\vec{x}^T\vec{c}\\
    \text{where  } \Vec{x}=\myvec{x_1\\x_2\\x_3\\x_4\\x_5} \text{ and } \vec{c}=\myvec{c_1\\c_2\\c_3\\c_4\\c_5}
\end{align}
Given three vectors
\begin{multline}
    \begin{aligned}
    &\alpha_1=\myvec{1\\2\\1\\0\\0}, &\alpha_2=\myvec{0\\1\\3\\3\\1},
    &\alpha_3=\myvec{1\\4\\6\\4\\1}
    \end{aligned}
    \end{multline}
Let matrix $A$ with column vectors $\alpha_1,\alpha_2,\alpha_3$:
\begin{align}
    \Vec{A}=\myvec{1&0&1\\2&1&4\\1&3&6\\0&3&4\\0&1&1}\eqlabel{eqA}
\end{align}
Given that $\vec{f}$ is a linear functional on $\mathbb{R}^5$, then $\vec{f}$ is in $\vec{W^0}$ if and only if,
\begin{align}
    f\brak{\alpha_i}=0, i=1,2,3\\
    \implies \vec{A}^T\vec{c}=\vec{0}\eqlabel{eqATc}
\end{align}
Converting the \eqref{eqATc} into system of equations, we have,
\begin{align}
    \myvec{1&2&1&0&0\\0&1&3&3&1\\1&4&6&4&1}\myvec{c_1\\c_2\\c_3\\c_4\\c_5}=0\eqlabel{eqmatrix}
\end{align}
Converting \eqref{eqmatrix} into row reduced echelon form,
\begin{align}
     \myvec{1&2&1&0&0\\0&1&3&3&1\\1&4&6&4&1}\xrightarrow[]{rref}\myvec{1&0&0&4&3\\0&1&0&-3&-2\\0&0&1&2&1}\eqlabel{eqrref}
\end{align}
From \eqref{eqrref}, we have,
\begin{align}
    c_1=-(4c_4+3c_5)\\
    c_2=(3c_4+2c_5)\\
    c_3=-(2c_4+c_5)
\end{align}
Therefore, general element of $\vec{W^0}$ is therefore,
\begin{multline}
\begin{aligned}
f(x_1,\dots,x_5)=-(4c_4+3c_5)x_1+(3_4+2c_5)x_2\\
-(2c_4+c_5)x_3+c_4x_4+c_5x_5 \eqlabel{eqf}
\end{aligned}
\end{multline}
Therefore, dimension of $\vec{W^0}$ is 2 and a basis $\{f_1,f_2\}$ can be obtained by putting $c_4=0,c_5=1$ and $c_4=1,c_5=0$ in \eqref{eqf}
\begin{align}
    f_1\brak{x_1,\dots,x_5}=-3x_1+2x_2-x_3+x_5\\
    f_2\brak{x_1,\dots,x_5}=-4x_1+3x_2-2x_3+x_4
\end{align}


\end{document}