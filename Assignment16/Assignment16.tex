\documentclass[journal,12pt]{IEEEtran}
\usepackage{longtable}
\usepackage{setspace}
\usepackage{gensymb}
\singlespacing
\usepackage[cmex10]{amsmath}
\newcommand\myemptypage{
	\null
	\thispagestyle{empty}
	\addtocounter{page}{-1}
	\newpage
}
\usepackage{amsthm}
\usepackage{mdframed}
\usepackage{mathrsfs}
\usepackage{txfonts}
\usepackage{stfloats}
\usepackage{bm}
\usepackage{cite}
\usepackage{cases}
\usepackage{subfig}

\usepackage{longtable}
\usepackage{multirow}

\usepackage{enumitem}
\usepackage{mathtools}
\usepackage{steinmetz}
\usepackage{tikz}
\usepackage{circuitikz}
\usepackage{verbatim}
\usepackage{tfrupee}
\usepackage[breaklinks=true]{hyperref}
\usepackage{graphicx}
\usepackage{tkz-euclide}

\usetikzlibrary{calc,math}
\usepackage{listings}
    \usepackage{color}                                            %%
    \usepackage{array}                                            %%
    \usepackage{longtable}                                        %%
    \usepackage{calc}                                             %%
    \usepackage{multirow}                                         %%
    \usepackage{hhline}                                           %%
    \usepackage{ifthen}                                           %%
    \usepackage{lscape}     
\usepackage{multicol}
\usepackage{chngcntr}

\DeclareMathOperator*{\Res}{Res}

\renewcommand\thesection{\arabic{section}}
\renewcommand\thesubsection{\thesection.\arabic{subsection}}
\renewcommand\thesubsubsection{\thesubsection.\arabic{subsubsection}}

\renewcommand\thesectiondis{\arabic{section}}
\renewcommand\thesubsectiondis{\thesectiondis.\arabic{subsection}}
\renewcommand\thesubsubsectiondis{\thesubsectiondis.\arabic{subsubsection}}


\hyphenation{op-tical net-works semi-conduc-tor}
\def\inputGnumericTable{}                                 %%

\lstset{
%language=C,
frame=single, 
breaklines=true,
columns=fullflexible
}
\begin{document}
\onecolumn

\newtheorem{theorem}{Theorem}[section]
\newtheorem{problem}{Problem}
\newtheorem{proposition}{Proposition}[section]
\newtheorem{lemma}{Lemma}[section]
\newtheorem{corollary}[theorem]{Corollary}
\newtheorem{example}{Example}[section]
\newtheorem{definition}[problem]{Definition}

\newcommand{\BEQA}{\begin{eqnarray}}
\newcommand{\EEQA}{\end{eqnarray}}
\newcommand{\define}{\stackrel{\triangle}{=}}
\bibliographystyle{IEEEtran}
\raggedbottom
\setlength{\parindent}{0pt}
\providecommand{\mbf}{\mathbf}
\providecommand{\pr}[1]{\ensuremath{\Pr\left(#1\right)}}
\providecommand{\qfunc}[1]{\ensuremath{Q\left(#1\right)}}
\providecommand{\sbrak}[1]{\ensuremath{{}\left[#1\right]}}
\providecommand{\lsbrak}[1]{\ensuremath{{}\left[#1\right.}}
\providecommand{\rsbrak}[1]{\ensuremath{{}\left.#1\right]}}
\providecommand{\brak}[1]{\ensuremath{\left(#1\right)}}
\providecommand{\lbrak}[1]{\ensuremath{\left(#1\right.}}
\providecommand{\rbrak}[1]{\ensuremath{\left.#1\right)}}
\providecommand{\cbrak}[1]{\ensuremath{\left\{#1\right\}}}
\providecommand{\lcbrak}[1]{\ensuremath{\left\{#1\right.}}
\providecommand{\rcbrak}[1]{\ensuremath{\left.#1\right\}}}
\theoremstyle{remark}
\newtheorem{rem}{Remark}
\newcommand{\sgn}{\mathop{\mathrm{sgn}}}
\providecommand{\abs}[1]{\left\vert#1\right\vert}
\providecommand{\res}[1]{\Res\displaylimits_{#1}} 
\providecommand{\norm}[1]{\left\lVert#1\right\rVert}
%\providecommand{\norm}[1]{\lVert#1\rVert}
\providecommand{\mtx}[1]{\mathbf{#1}}
\providecommand{\mean}[1]{E\left[ #1 \right]}
\providecommand{\fourier}{\overset{\mathcal{F}}{ \rightleftharpoons}}
%\providecommand{\hilbert}{\overset{\mathcal{H}}{ \rightleftharpoons}}
\providecommand{\system}{\overset{\mathcal{H}}{ \longleftrightarrow}}
	%\newcommand{\solution}[2]{\textbf{Solution:}{#1}}
\newcommand{\solution}{\noindent \textbf{Solution: }}
\newcommand{\cosec}{\,\text{cosec}\,}
\providecommand{\dec}[2]{\ensuremath{\overset{#1}{\underset{#2}{\gtrless}}}}
\newcommand{\myvec}[1]{\ensuremath{\begin{pmatrix}#1\end{pmatrix}}}
\newcommand{\mydet}[1]{\ensuremath{\begin{vmatrix}#1\end{vmatrix}}}
\numberwithin{equation}{subsection}
\makeatletter
\@addtoreset{figure}{problem}
\makeatother
\let\StandardTheFigure\thefigure
\let\vec\mathbf
\renewcommand{\thefigure}{\theproblem}
\def\putbox#1#2#3{\makebox[0in][l]{\makebox[#1][l]{}\raisebox{\baselineskip}[0in][0in]{\raisebox{#2}[0in][0in]{#3}}}}
     \def\rightbox#1{\makebox[0in][r]{#1}}
     \def\centbox#1{\makebox[0in]{#1}}
     \def\topbox#1{\raisebox{-\baselineskip}[0in][0in]{#1}}
     \def\midbox#1{\raisebox{-0.5\baselineskip}[0in][0in]{#1}}
\vspace{3cm}
\title{Assignment 16}
\author{Rubeena Aafreen}
\maketitle
\bigskip
\renewcommand{\thefigure}{\theenumi}
\renewcommand{\thetable}{\theenumi}
%
Download the latex-tikz codes from 
%
\begin{lstlisting}
https://github.com/rubeenaafreen20/EE5609/tree/master/Assignment16
\end{lstlisting}
\section{\textbf{Problem}}
%
True or False? If a diagonalizable operator has only the characteristic values $0$ and $1$, it is a projection.
\section{\textbf{Definitions}}
\renewcommand{\thetable}{1}
\begin{table}[ht!]
\centering
\begin{tabular}{|c|l|}
    \hline
    \multirow{3}{*}{Diagonalizable Operator} 
	& \\
	& For a linear operator $\vec{T}\colon \vec{V}\longrightarrow \vec{V}$, $\vec{T}$ is a diagonalizable operator if \\
	& $\exists$ some basis for $\Vec{V}$ such that the matrix representing T is a diagonal matrix\\
	&i.e.\\
	& $\qquad\qquad\qquad \vec{T}\brak{\vec{X}}=\Vec{A}\Vec{X}$,\\
    &$\implies \vec{A}$ is a diagonalizable matrix\\
	&\\
	\hline
	\multirow{3}{*}{Properties of Projection} 
	& \\
	& If $n\times n$ matrix $\vec{A}$ is projecion matrix, then\\
	&\\
	&$\qquad\qquad\qquad \vec{A}^2=\vec{A}$\\
	&\\
	\hline
\end{tabular}
\label{table:1}
    \caption{Definitions}
\end{table}
\newpage
\section{\textbf{Solution}}
\renewcommand{\thetable}{2}
\begin{longtable}{|c|l|}
    \hline
	\multirow{3}{*}{Diagonalizability} 
	& \\
	& Let $\vec{A}$ be $n\times n$ matrix.\\
	& Given that $\vec{A}$ is diagonalizable, it can be expressed as,\\
	&\\
	& $\qquad \qquad\qquad \vec{A}=\vec{P}\vec{\Lambda}\vec{P}^{-1}$\\
	& $\qquad\qquad\implies \vec{A}\vec{P}=\vec{P}\vec{\Lambda}\qquad\qquad\qquad\qquad\qquad\dots\brak{1}$\\
	&\\
	& where, $\vec{\Lambda}=\myvec{\lambda_1&0&\dots&0\\0&\lambda_2&\dots&0\\ \vdots&\vdots&\dots&\vdots\\0&0&\dots&\lambda_n}$\\
	&\\
	\hline
	\multirow{3}{*}{Eigen values} & \\
	& Given that $\vec{A}$ has eigen values $0$ and $1$\\
	&$\implies \vec{\Lambda}$ has diagonal entries of $0s$ and $1s$ only\\ 
	&\\
	&$\implies \lambda_i=0$ or $1,\qquad i=0,1,\dots,n$\\
	&$\implies \lambda_i^2=0$ or $1$\\
	&$\implies \lambda_i^2=\lambda_i$\\
	&\\
	& $\qquad\implies \Vec{\Lambda}^2=\Vec{\Lambda}\Vec{\Lambda}=\myvec{\lambda_1^2&0&\dots&0\\0&\lambda_2^2&\dots&0\\ \vdots&\vdots&\dots&\vdots\\0&0&\dots&\lambda_n^2}=\vec{\Lambda}\qquad\dots\brak{2}$\\
	&\\
	\hline
	\multirow{3}{*}{Projection}&\\ 
	&$\qquad\qquad\vec{A}=\vec{P}\vec{\Lambda}\vec{P}^{-1}$\\
	&$\qquad\implies \vec{A}\vec{A}=\vec{A}\vec{P}\vec{\Lambda}\vec{P}^{-1}$\\
	&$\qquad\implies \vec{A}^2=\brak{\vec{A}\vec{P}}\vec{\Lambda}\vec{P}^{-1}$\\
	&\\
	&From $\brak{1}$,\\
	&$\qquad\implies \vec{A}^2=\vec{P}\vec{\Lambda}\vec{\Lambda}\vec{P}^{-1}$\\
	&$\qquad\qquad\qquad=\vec{P}\vec{\Lambda}^2\vec{P}^{-1}$\\
	&From $\brak{2}$,\\
	&$\qquad\implies \vec{A}^2=\vec{P}\vec{\Lambda}\vec{P}^{-1}$\\
	&$\qquad\qquad\qquad=\vec{A}$\\
	&\\
	&Therefore,\\
	&$\qquad\qquad\qquad\vec{A}^2=\vec{A}$\\
	&\\
	& Hence, $\vec{A}$ is a projection matrix\\
	&\\
	\hline
	\multirow{3}{*}{Conclusion} & \\
	& Given statement is True\\
	&\\
	\hline
	\caption{Checking for projection matrix}
    \label{table:2}
\end{longtable}
\end{document}
