\documentclass[journal,12pt]{IEEEtran}
\usepackage{longtable}
\usepackage{setspace}
\usepackage{gensymb}
\singlespacing
\usepackage[cmex10]{amsmath}
\newcommand\myemptypage{
	\null
	\thispagestyle{empty}
	\addtocounter{page}{-1}
	\newpage
}
\usepackage{amsthm}
\usepackage{mdframed}
\usepackage{mathrsfs}
\usepackage{txfonts}
\usepackage{stfloats}
\usepackage{bm}
\usepackage{cite}
\usepackage{cases}
\usepackage{subfig}

\usepackage{longtable}
\usepackage{multirow}

\usepackage{enumitem}
\usepackage{mathtools}
\usepackage{steinmetz}
\usepackage{tikz}
\usepackage{circuitikz}
\usepackage{verbatim}
\usepackage{tfrupee}
\usepackage[breaklinks=true]{hyperref}
\usepackage{graphicx}
\usepackage{tkz-euclide}

\usetikzlibrary{calc,math}
\usepackage{listings}
    \usepackage{color}                                            %%
    \usepackage{array}                                            %%
    \usepackage{longtable}                                        %%
    \usepackage{calc}                                             %%
    \usepackage{multirow}                                         %%
    \usepackage{hhline}                                           %%
    \usepackage{ifthen}                                           %%
    \usepackage{lscape}     
\usepackage{multicol}
\usepackage{chngcntr}

\DeclareMathOperator*{\Res}{Res}

\renewcommand\thesection{\arabic{section}}
\renewcommand\thesubsection{\thesection.\arabic{subsection}}
\renewcommand\thesubsubsection{\thesubsection.\arabic{subsubsection}}

\renewcommand\thesectiondis{\arabic{section}}
\renewcommand\thesubsectiondis{\thesectiondis.\arabic{subsection}}
\renewcommand\thesubsubsectiondis{\thesubsectiondis.\arabic{subsubsection}}


\hyphenation{op-tical net-works semi-conduc-tor}
\def\inputGnumericTable{}                                 %%

\lstset{
%language=C,
frame=single, 
breaklines=true,
columns=fullflexible
}
\begin{document}
\onecolumn

\newtheorem{theorem}{Theorem}[section]
\newtheorem{problem}{Problem}
\newtheorem{proposition}{Proposition}[section]
\newtheorem{lemma}{Lemma}[section]
\newtheorem{corollary}[theorem]{Corollary}
\newtheorem{example}{Example}[section]
\newtheorem{definition}[problem]{Definition}

\newcommand{\BEQA}{\begin{eqnarray}}
\newcommand{\EEQA}{\end{eqnarray}}
\newcommand{\define}{\stackrel{\triangle}{=}}
\bibliographystyle{IEEEtran}
\raggedbottom
\setlength{\parindent}{0pt}
\providecommand{\mbf}{\mathbf}
\providecommand{\pr}[1]{\ensuremath{\Pr\left(#1\right)}}
\providecommand{\qfunc}[1]{\ensuremath{Q\left(#1\right)}}
\providecommand{\sbrak}[1]{\ensuremath{{}\left[#1\right]}}
\providecommand{\lsbrak}[1]{\ensuremath{{}\left[#1\right.}}
\providecommand{\rsbrak}[1]{\ensuremath{{}\left.#1\right]}}
\providecommand{\brak}[1]{\ensuremath{\left(#1\right)}}
\providecommand{\lbrak}[1]{\ensuremath{\left(#1\right.}}
\providecommand{\rbrak}[1]{\ensuremath{\left.#1\right)}}
\providecommand{\cbrak}[1]{\ensuremath{\left\{#1\right\}}}
\providecommand{\lcbrak}[1]{\ensuremath{\left\{#1\right.}}
\providecommand{\rcbrak}[1]{\ensuremath{\left.#1\right\}}}
\theoremstyle{remark}
\newtheorem{rem}{Remark}
\newcommand{\sgn}{\mathop{\mathrm{sgn}}}
\providecommand{\abs}[1]{\left\vert#1\right\vert}
\providecommand{\res}[1]{\Res\displaylimits_{#1}} 
\providecommand{\norm}[1]{\left\lVert#1\right\rVert}
%\providecommand{\norm}[1]{\lVert#1\rVert}
\providecommand{\mtx}[1]{\mathbf{#1}}
\providecommand{\mean}[1]{E\left[ #1 \right]}
\providecommand{\fourier}{\overset{\mathcal{F}}{ \rightleftharpoons}}
%\providecommand{\hilbert}{\overset{\mathcal{H}}{ \rightleftharpoons}}
\providecommand{\system}{\overset{\mathcal{H}}{ \longleftrightarrow}}
	%\newcommand{\solution}[2]{\textbf{Solution:}{#1}}
\newcommand{\solution}{\noindent \textbf{Solution: }}
\newcommand{\cosec}{\,\text{cosec}\,}
\providecommand{\dec}[2]{\ensuremath{\overset{#1}{\underset{#2}{\gtrless}}}}
\newcommand{\myvec}[1]{\ensuremath{\begin{pmatrix}#1\end{pmatrix}}}
\newcommand{\mydet}[1]{\ensuremath{\begin{vmatrix}#1\end{vmatrix}}}
\numberwithin{equation}{subsection}
\makeatletter
\@addtoreset{figure}{problem}
\makeatother
\let\StandardTheFigure\thefigure
\let\vec\mathbf
\renewcommand{\thefigure}{\theproblem}
\def\putbox#1#2#3{\makebox[0in][l]{\makebox[#1][l]{}\raisebox{\baselineskip}[0in][0in]{\raisebox{#2}[0in][0in]{#3}}}}
     \def\rightbox#1{\makebox[0in][r]{#1}}
     \def\centbox#1{\makebox[0in]{#1}}
     \def\topbox#1{\raisebox{-\baselineskip}[0in][0in]{#1}}
     \def\midbox#1{\raisebox{-0.5\baselineskip}[0in][0in]{#1}}
\vspace{3cm}
\title{Assignment 14}
\author{Rubeena Aafreen}
\maketitle
\bigskip
\renewcommand{\thefigure}{\theenumi}
\renewcommand{\thetable}{\theenumi}
%
Download the latex-tikz codes from 
%
\begin{lstlisting}
https://github.com/rubeenaafreen20/EE5609/tree/master/Assignment14
\end{lstlisting}
\section{\textbf{Problem}}
%
Let $\Vec{A}$ be the $4\times 4$ real matrix
\begin{align}
    \Vec{A}=\myvec{1&1&0&0\\-1&-1&0&0\\-2&-2&2&1\\1&1&-1&0}
\end{align}
Show that the characteristic polynomial for $\Vec{A}$ is $x^2\brak{x-1}^2$ and that it is also the minimal polynomial
\section{\textbf{Definitions}}
\renewcommand{\thetable}{1}
\begin{table}[ht!]
\centering
\begin{tabular}{|c|l|}
    \hline
	\multirow{3}{*}{Characteristic Polynomial} 
	& \\
	& For an $n\times n$ matrix $\vec{A}$, characteristic polynomial is defined by,\\
	&\\
	& $\qquad\qquad\qquad p\brak{x}=\mydet{x\Vec{I}-\Vec{A}}$\\
	&\\
	\hline
	\multirow{3}{*}{Minimal Polynomial} 
	&\\
	& Minimal polynomial $m\brak{x}$ is he smallest factor of characteristic polynomial\\
	& $p\brak{x}$ such that,\\
	&\\
	& $\qquad \qquad \qquad m\brak{\vec{A}}=0$\\
	&\\
    \hline
    \multirow{3}{*}{Algebraic Multiplicity $\brak{A_M}$}
    &\\
    & No. of times an eigen value appears in a characteristic equation.\\
    &\\
    \hline
\end{tabular}
\label{table:1}
    \caption{Definitions}
\end{table}
\newpage
\section{\textbf{Solution}}
\renewcommand{\thetable}{2}
\begin{longtable}{|c|l|}
    \hline
	\multirow{3}{*}{Characteristic polynomial} 
	& \\
	& $p\brak{x}=\mydet{x\Vec{I}-\Vec{A}}$\\
	& $\qquad = \mydet{x-1&-1\\1&x+1}\mydet{x-2&-1\\1&x}$\\
	& $\qquad=\brak{\brak{x-1}\brak{x+1}+1}\brak{\brak{x-2}x+1}$\\
	&$\qquad=x^2\brak{x^2-2x+1}$\\
	&$\qquad=x^2\brak{x-1}^2$\\
	&\\
	\hline
	\multirow{3}{*}{$A_M$} 
	&\\
	& For $\lambda=0,\quad A_M=2$\\
	& For $\lambda=1,\quad A_M=2$\\
	&\\
	\hline
	\multirow{3}{*}{Minimal Polynomial} & \\
	& $p\brak{x}=x^a\brak{x-1}^b,\quad a\leq2, b\leq2$\\
	&\\
	\hline
	\multirow{3}{*}{$a=1,b=1$} & \\
	&$m\brak{x}=x\brak{x-1}$\\
	&\\
	& $\implies m\brak{\vec{A}}=\vec{A}\brak{\Vec{A}-\vec{I}}$\\
	&\\
	& $\qquad\qquad=\myvec{1&1&0&0\\-1&-1&0&0\\-2&-2&2&1\\1&1&-1&0}\myvec{0&1&0&0\\-1&-2&0&0\\-2&-2&1&1\\1&1&-1&-1}$\\
	& $\qquad \qquad = \myvec{-1&-1&0&0\\1&1&0&0\\-1&-1&1&1\\1&1&-2&-2}\neq\Vec{0}$\\
	&\\
	& $\implies x\brak{x-1}$ is not a minimal polynomial\\
	&\\
	\hline
	\multirow{3}{*}{$a=2,b=1$} & \\
	&$m\brak{x}=x^2\brak{x-1}$\\
	&\\
	& $\implies m\brak{\vec{A}}=\vec{A}^2\brak{\Vec{A}-\vec{I}}$\\
	&\\
	& $\qquad \qquad=\myvec{0&0&0&0\\0&0&0&0\\-3&-3&3&2\\2&2&-2&-1}\myvec{0&1&0&0\\-1&-2&0&0\\-2&-2&1&1\\1&1&-1&-1}$\\
	& $\qquad \qquad = \myvec{0&0&0&0\\0&0&0&0\\-1&-1&1&1\\1&1&-1&-1}\neq\Vec{0}$\\
	&\\
	&$\implies x^2\brak{x-1}$ is not a minimal polynomial\\
	&\\
	\hline
	\multirow{3}{*}{$a=1,b=2$} & \\
	&$m\brak{x}=x\brak{x-1}^2$\\
	&\\
	& $\implies m\brak{\vec{A}}=\vec{A}\brak{\Vec{A}-\vec{I}}^2$\\
	&\\
	& $\qquad\qquad=\myvec{1&1&0&0\\-1&-1&0&0\\-2&-2&2&1\\1&1&-1&0}\myvec{1&-2&0&0\\2&3&0&0\\1&1&0&0\\0&0&0&0}$\\
	& $\qquad\qquad = \myvec{1&1&0&0\\-1&-1&0&0\\0&0&0&0\\0&0&0&0}\neq\Vec{0}$\\
	&\\
	&$\implies x^2\brak{x-1}$ is not a minimal polynomial\\
	&\\
	\hline
	\multirow{3}{*}{$a=2,b=2$} & \\
	&$m\brak{x}=x^2\brak{x-1}^2$\\
	&\\
	&$\implies m\brak{\vec{A}}=\vec{A}^2\brak{\Vec{A}-\vec{I}}^2$\\
	&\\
	& $\qquad\qquad=\myvec{0&0&0&0\\0&0&0&0\\-3&-3&3&2\\2&2&-2&-1}\myvec{1&-2&0&0\\2&3&0&0\\1&1&0&0\\0&0&0&0}$\\
	& $\qquad\qquad = \myvec{0&0&0&0\\0&0&0&0\\0&0&0&0\\0&0&0&0}=\Vec{0}$\\
	&\\
	&$\implies x^2\brak{x-1}^2$ is a minimal polynomial\\
	&\\
	\hline
	\multirow{3}{*}{Conclusion} & \\
	& For the given matrix $\Vec{A}$,\\
	&$x^2\brak{x-1}^2$ is the characteristic polynomial as well as minimal polynomial.\\
	&\\
	\hline
	\caption{Proving that the given characteristic polynomial is also minimal polynomial}
    \label{table:2}
\end{longtable}
\end{document}
