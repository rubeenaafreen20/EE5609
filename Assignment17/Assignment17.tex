\documentclass[journal,12pt]{IEEEtran}
\usepackage{longtable}
\usepackage{setspace}
\usepackage{gensymb}
\singlespacing
\usepackage[cmex10]{amsmath}
\newcommand\myemptypage{
	\null
	\thispagestyle{empty}
	\addtocounter{page}{-1}
	\newpage
}
\usepackage{amsthm}
\usepackage{mdframed}
\usepackage{mathrsfs}
\usepackage{txfonts}
\usepackage{stfloats}
\usepackage{bm}
\usepackage{cite}
\usepackage{cases}
\usepackage{subfig}

\usepackage{longtable}
\usepackage{multirow}

\usepackage{enumitem}
\usepackage{mathtools}
\usepackage{steinmetz}
\usepackage{tikz}
\usepackage{circuitikz}
\usepackage{verbatim}
\usepackage{tfrupee}
\usepackage[breaklinks=true]{hyperref}
\usepackage{graphicx}
\usepackage{tkz-euclide}

\usetikzlibrary{calc,math}
\usepackage{listings}
    \usepackage{color}                                            %%
    \usepackage{array}                                            %%
    \usepackage{longtable}                                        %%
    \usepackage{calc}                                             %%
    \usepackage{multirow}                                         %%
    \usepackage{hhline}                                           %%
    \usepackage{ifthen}                                           %%
    \usepackage{lscape}     
\usepackage{multicol}
\usepackage{chngcntr}

\DeclareMathOperator*{\Res}{Res}

\renewcommand\thesection{\arabic{section}}
\renewcommand\thesubsection{\thesection.\arabic{subsection}}
\renewcommand\thesubsubsection{\thesubsection.\arabic{subsubsection}}

\renewcommand\thesectiondis{\arabic{section}}
\renewcommand\thesubsectiondis{\thesectiondis.\arabic{subsection}}
\renewcommand\thesubsubsectiondis{\thesubsectiondis.\arabic{subsubsection}}


\hyphenation{op-tical net-works semi-conduc-tor}
\def\inputGnumericTable{}                                 %%

\lstset{
%language=C,
frame=single, 
breaklines=true,
columns=fullflexible
}
\begin{document}
\onecolumn

\newtheorem{theorem}{Theorem}[section]
\newtheorem{problem}{Problem}
\newtheorem{proposition}{Proposition}[section]
\newtheorem{lemma}{Lemma}[section]
\newtheorem{corollary}[theorem]{Corollary}
\newtheorem{example}{Example}[section]
\newtheorem{definition}[problem]{Definition}

\newcommand{\BEQA}{\begin{eqnarray}}
\newcommand{\EEQA}{\end{eqnarray}}
\newcommand{\define}{\stackrel{\triangle}{=}}
\bibliographystyle{IEEEtran}
\raggedbottom
\setlength{\parindent}{0pt}
\providecommand{\mbf}{\mathbf}
\providecommand{\pr}[1]{\ensuremath{\Pr\left(#1\right)}}
\providecommand{\qfunc}[1]{\ensuremath{Q\left(#1\right)}}
\providecommand{\sbrak}[1]{\ensuremath{{}\left[#1\right]}}
\providecommand{\lsbrak}[1]{\ensuremath{{}\left[#1\right.}}
\providecommand{\rsbrak}[1]{\ensuremath{{}\left.#1\right]}}
\providecommand{\brak}[1]{\ensuremath{\left(#1\right)}}
\providecommand{\lbrak}[1]{\ensuremath{\left(#1\right.}}
\providecommand{\rbrak}[1]{\ensuremath{\left.#1\right)}}
\providecommand{\cbrak}[1]{\ensuremath{\left\{#1\right\}}}
\providecommand{\lcbrak}[1]{\ensuremath{\left\{#1\right.}}
\providecommand{\rcbrak}[1]{\ensuremath{\left.#1\right\}}}
\theoremstyle{remark}
\newtheorem{rem}{Remark}
\newcommand{\sgn}{\mathop{\mathrm{sgn}}}
\providecommand{\abs}[1]{\left\vert#1\right\vert}
\providecommand{\res}[1]{\Res\displaylimits_{#1}} 
\providecommand{\norm}[1]{\left\lVert#1\right\rVert}
%\providecommand{\norm}[1]{\lVert#1\rVert}
\providecommand{\mtx}[1]{\mathbf{#1}}
\providecommand{\mean}[1]{E\left[ #1 \right]}
\providecommand{\fourier}{\overset{\mathcal{F}}{ \rightleftharpoons}}
%\providecommand{\hilbert}{\overset{\mathcal{H}}{ \rightleftharpoons}}
\providecommand{\system}{\overset{\mathcal{H}}{ \longleftrightarrow}}
	%\newcommand{\solution}[2]{\textbf{Solution:}{#1}}
\newcommand{\solution}{\noindent \textbf{Solution: }}
\newcommand{\cosec}{\,\text{cosec}\,}
\providecommand{\dec}[2]{\ensuremath{\overset{#1}{\underset{#2}{\gtrless}}}}
\newcommand{\myvec}[1]{\ensuremath{\begin{pmatrix}#1\end{pmatrix}}}
\newcommand{\mydet}[1]{\ensuremath{\begin{vmatrix}#1\end{vmatrix}}}
\numberwithin{equation}{subsection}
\makeatletter
\@addtoreset{figure}{problem}
\makeatother
\let\StandardTheFigure\thefigure
\let\vec\mathbf
\renewcommand{\thefigure}{\theproblem}
\def\putbox#1#2#3{\makebox[0in][l]{\makebox[#1][l]{}\raisebox{\baselineskip}[0in][0in]{\raisebox{#2}[0in][0in]{#3}}}}
     \def\rightbox#1{\makebox[0in][r]{#1}}
     \def\centbox#1{\makebox[0in]{#1}}
     \def\topbox#1{\raisebox{-\baselineskip}[0in][0in]{#1}}
     \def\midbox#1{\raisebox{-0.5\baselineskip}[0in][0in]{#1}}
\vspace{3cm}
\title{Assignment 17}
\author{Rubeena Aafreen}
\maketitle
\bigskip
\renewcommand{\thefigure}{\theenumi}
\renewcommand{\thetable}{\theenumi}
%
Download the latex-tikz codes from 
%
\begin{lstlisting}
https://github.com/rubeenaafreen20/EE5609/tree/master/Assignment17
\end{lstlisting}
\section{\textbf{Problem}}
%
Let $\vec{T}$ be the diagonalizable linear operator on $\mathbb{R}^3$ which we discussed in example 3 of section $6.2$. Use the Lagrange polynomials to write the representing matrix $\vec{A}$ in the form
\begin{align}
    \vec{A}=\vec{E_1}+2\vec{E_2},\quad \vec{E_1}+\vec{E_2}=\vec{I}, \vec{E_1}\vec{E_2}=\vec{0}
\end{align}
\section{\textbf{Outline}}
\renewcommand{\thetable}{1}
\begin{longtable}{|c|l|}
    \hline
    \multirow{3}{*}{Diagonalizable Operator} 
	& \\
	& For a linear operator $\vec{T}\colon \vec{V}\longrightarrow \vec{V}$, $\vec{T}$ is a diagonalizable operator if \\
	& $\exists$ some basis for $\Vec{V}$ such that the matrix representing $\vec{T}$ is a diagonal matrix\\
	&i.e.\\
	& $\qquad\qquad\qquad \vec{T}\brak{\vec{X}}=\Vec{A}\Vec{X}$,\\
    &$\implies \vec{A}$ is a diagonalizable matrix\\
	&\\
	\hline
	\multirow{3}{*}{Characteristic Polynomial} 
	& \\
	& For an $n\times n$ matrix $\vec{A}$, characteristic polynomial is defined by,\\
	&\\
	& $\qquad\qquad\qquad p\brak{x}=\mydet{x\Vec{I}-\Vec{A}}$\\
	&\\
	\hline
	\multirow{3}{*}{Minimal Polynomial} 
	&\\
	& Minimal polynomial $m\brak{x}$ is the smallest factor of characteristic polynomial\\
	& $p\brak{x}$ such that,\\
	&\\
	& $\qquad \qquad \qquad m\brak{\vec{A}}=0$\\
	& \\
	& Every root of characteristic polynomial should be the root of minimal\\
	& polynomial\\
	&\\
    \hline
	\multirow{3}{*}{Lagrange Polynomials} 
	& \\
	& For a set of scalars $c_0, c_1,\dots, c_n \in \mathbb{F}$, Lagrange Polynomial is defined as:\\
	&\\
	&$\qquad\qquad\qquad p_j=\displaystyle\prod_{i\neq j}\frac{\brak{x-c_i}}{\brak{c_j-c_i}}$\\
	&\\
	\hline
	\multirow{3}{*}{Theorem} 
	& \\
	& If $\vec{T}$ is a diagonalizable linear operator on a finite dimensional space $\vec{V}$,\\
	&and if $c_1,c_2,\dots,c_k$ are distinct characterictic values of $\vec{T}$, then there exist\\
	&linear operators $\vec{E_1},\vec{E_2},\dots,\vec{E_k}$ such that:\\
	&\\
	& $\quad\brak{1}\quad\vec{T}=c_1\vec{E_1}+\dots+c_k\vec{E_k}$\\
	&$\quad\brak{2}\quad\vec{E_1}+\dots+\vec{E_k}=\vec{I}$\\
	&$\quad\brak{3}\quad\vec{E_i}\vec{E_j}=\vec{0},\quad i\neq j$\\
	&$\quad\brak{4}\quad\vec{E_i}=\vec{E_i}^2,\quad\brak{\vec{E_i}\text{ is a projection}}$\\
	&$\quad\brak{5}\quad\alpha=\vec{E_i}\alpha,\forall\alpha\in\vec{V}$\\
	&\\
	\hline
	\multirow{3}{*}{\shortstack{Relation between Lagrange\\ Polynomials and Projection}} 
	& \\
	& We have:\\
	&$\qquad\qquad \vec{T}=c_1\vec{E_1}+\dots+c_k\vec{E_k}$\\
	&\\
	&If $g$ is any polynomial over field $\mathbb{F}$,\\ 
	&$\qquad\qquad g\brak{\vec{T}}=g\brak{c_1}\vec{E_1}+\dots+g\brak{c_k}\vec{E_k}\qquad\qquad\qquad\dots\brak{1}$\\
	&\\
	&Now,\\
	&$\qquad\qquad p_j=\displaystyle\prod_{i\neq j}\frac{\brak{x-c_i}}{\brak{c_j-c_i}}$\\
	&$\qquad\implies p_j\brak{c_i}=\delta_{ij}$ (Kronecker Delta)$\qquad\qquad\qquad\quad\dots\brak{2}$\\
	&\\
	&From $\brak{1}$ and $\brak{2}$,\\
	&$\qquad\implies p_j\brak{\vec{T}}=\sum_{i=1}^k\delta_{ij}\vec{E}_i=\vec{E_j}$\\
	&\\
	&$\qquad\qquad\implies\boxed{p_j\brak{\vec{T}}=\vec{E_j}}$\\
	&\\
	&$\implies$ Projections $\vec{E_j}$ are polynomials in $\vec{T}$\\
	&\\
	\hline
    \caption{Definitions and results used}
    \label{table:1}
\end{longtable}
\section{\textbf{Solution}}
\renewcommand{\thetable}{2}
\begin{longtable}{|c|l|}
    \hline
    \multirow{3}{*}{Given} 
	& \\
	&Matrix of $\vec{T}$ in the standard basis of $\mathbb{R}^3:$\\
	& $\qquad\qquad\qquad\vec{A}=\myvec{5&-6&-6\\-1&4&2\\3&-6&-4}$\\
	&\\
	\hline
	\multirow{3}{*}{Characteristic polynomial} 
	& \\
	& $p\brak{x}=\mydet{x\Vec{I}-\Vec{A}}$\\
	&\\
	& $\qquad = \mydet{x&-1&0\\-2&x+2&-2\\-2&3&x-2}$\\
	& $\qquad=x^3-5x^2+8x-4$\\
	&$\qquad=\brak{x-1}\brak{x-2}^2$\\
	&\\
	&$\implies \lambda=1, 2$\\
	&\\
	\hline
	\multirow{3}{*}{Minimal Polynomial} & \\
	& $p\brak{x}=\brak{x-1}\brak{x-2}^b,\quad b\leq2$\\
	&\\
	&$\brak{\vec{A}-\vec{I}}\brak{\vec{A}-2\vec{I}}=\myvec{4&-6&-6\\-1&3&2\\3&-6&-5}\myvec{3&-6&-6\\-1&2&2\\3&-6&-6}=\vec{0}$\\
	&\\
	&Therefore, $\brak{x-1}\brak{x-2}$ is the minimal polynomial.\\
	&\\
	\hline
	\multirow{3}{*}{Lagrange Polynomial} 
	& \\
	& $\qquad\qquad p_j=\displaystyle\prod_{i\neq j} \frac{\brak{x-c_i}}{\brak{c_j-c_i}}$\\
	&\\
	& For characteristic values $c_1=1,\quad c_2=2$,\\
	&\\
	& $\implies p_1=\frac{(x-1)}{2-1}, \qquad p_2=\frac{(x-2)}{1-2}$\\
	&\\
	& $\implies p_1= \brak{x-1}$, and\\
	&$\qquad p_2=\brak{2-x}$\\
	&\\
	\hline
	\multirow{3}{*}{Projection Maps} & \\
	& We know that,\\
	&$\qquad \qquad \qquad \vec{E_j}=p_j\brak{\vec{T
	}}$\\ 
	&\\
	&$\implies \vec{E_1}=\vec{A}-\vec{I}$ and $\vec{E_2}=2\vec{I}-\vec{A}$ \\
	&\\
	&$\implies \vec{E_1}=\myvec{4&-6&-6\\-1&3&2\\3&-6&-5}$, and \\
	&$\qquad \vec{E_2}=\myvec{-3&6&6\\1&-2&-2\\-3&6&6}$\\
	&\\
	\hline
\multirow{3}{*}{Verification} & \\
	& We have, \\
	&$\qquad\qquad\vec{E_1}=\vec{A}-\vec{I}$\\
	&$\qquad\implies \vec{A}-\vec{E_1}=\vec{I}\qquad\qquad\qquad\dots\brak{1}$\\
	&\\
	&$\qquad\qquad\vec{E_2}=2\vec{I}-\vec{A}$\\
	&From $\brak{1},$\\
	&$\qquad\implies\vec{E_2}=2\brak{\vec{A}-\vec{E_1}}-\vec{A}$\\
	&$\qquad\implies \boxed{\vec{A}=2\vec{E_1}+\vec{E_2}}$\\
	&\\
	& Also,\\
	&$\qquad \vec{E_1}=\myvec{4&-6&-6\\-1&3&2\\3&-6&-5},\qquad \vec{E_2}=\myvec{-3&6&6\\1&-2&-2\\-3&6&6}$\\
	&\\
	&$\implies \vec{E_1}+\vec{E_2}=\myvec{4&-6&-6\\-1&3&2\\3&-6&-5}+\myvec{-3&6&6\\1&-2&-2\\-3&6&6}$\\
	&$\qquad\qquad\qquad = \myvec{1&0&0\\0&1&0\\0&0&1}$\\
	&\\
	&$\qquad\implies\boxed{\vec{E_1}+\vec{E_2}=\vec{I}}$\\
	&\\
	&$\vec{E_1}\vec{E_2}=\myvec{4&-6&-6\\-1&3&2\\3&-6&-5}\myvec{-3&6&6\\1&-2&-2\\-3&6&6}$\\
	&$\qquad\quad=\myvec{0&0&0\\0&0&0\\0&0&0}$\\
	&\\
	&$\qquad\implies\boxed{\vec{E_1}\vec{E_2}=\vec{0}}$\\
	&\\
	\hline
	\caption{Using Lagrange Polynomials to represent $\vec{A}$}
    \label{table:2}
\end{longtable}
\end{document}
